\documentclass[11pt]{article}

\usepackage{vmargin}
\setmarginsrb{0.5in}{0.5in}{0.5in}{0.5in}{20pt}{20pt}{20pt}{20pt}

\title{The Hoffman Reference Guide}
\author{Brent Baccala}

\usepackage{times}
\pagestyle{empty}

% Get rid of page numbers
\def\thepage{}

\begin{document}

\maketitle

\parindent 0pt
\parskip 12pt

Hoffman is a program to solve chess endgames using retrograde
analysis, which is much different from conventional computer chess
programs.  Retrograde analysis is only useful in the endgame, runs
very slowly, and produces enormous amounts of data.  Its great
advantage lies in its ability to completely solve the endgame.  In a
very real sense, a retrograde engine has no ``move horizon'' like a
conventional chess engine.  It sees everything.  For those not up on
Americana, the program is named after Trevor Hoffman, an All Star
baseball pitcher who specializes in ``closing'' games.  It was written
specifically for The World vs. Arno Nickel game.

Hoffman uses XML extensively both for configuring its operation and
for labeling the resulting tablebases.  In fact, a completed Hoffman
tablebase (typically in an {\tt .htb} file) is just a {\tt gzip}-ed
file that contains an XML prefix followed by binary tablebase data (in
a format specified by the XML).  Basically, to operate Hoffman, you
write an XML file that specifies the analysis you want done, then feed
it to the program.  As output, it produces a modified version of the
XML input that includes binary tablebase data appended at the end.

\section{Parallel Processing with Hoffman}

A Hoffman analysis can be quite compute-intensive.  The program can be
compiled to use POSIX threads (if available), with the number of
threads specified at run-time using the {\tt -t} option.  The program
is also designed to use multiple computers in parallel, all working
simultaneously on an analysis.  This is accomplished by breaking the
analysis up into smaller pieces, each with its own XML configuration
file.

%The primary support provided by the program is the ability to
%use URLs instead of filenames to reference tablebases, allowing
%tablebases to be stored on a server accessable over a network.  A
%basic Perl CGI script ({\tt hoffman.cgi}) is provided which, when
%installed on a web server and supplied with a directory full of
%Hoffman XML files, will hand them out in the proper order to Hoffman
%clients for processing.

\section{Propagation tables}

A Hoffman analysis can also be quite space-intensive.  Since its
memory utilization pattern is basically random, Hoffman will begin to
swap dramatically and suffer a disastrous drop in performance once its
working set size exceeds the machine's available memory.  To alleviate
this, the program can be operated in a mode where it fills a series of
{\it propagation tables}, writing each one out to disk when full, then
reads them back in sequentially during the next pass.  Although less
efficient than when the working set can be contained in memory,
propagation tables allow the program to build tablebases of
essentially unlimited size with no swapping and reasonable CPU
utilization.  This mode is activated at run-time by specifying the
size of the propagation tables (in MB) with the {\tt -P} switch.
Temporary files will be written to the current directory.

For example, the command {\tt hoffman -g -t 2 -P 1024 kqqkqq.xml} will
trigger a Hoffman generation run with two threads, using one gigabyte
(1024 megabytes) of memory.

\vfil\eject
\section{XML Syntax}

The root XML element in a Hoffman tablebase is always {\tt
<tablebase>}.  Its only attribute ({\tt offset}) is added by the
program, should not be supplied by the user, and indicates a
hexadecimal byte-offset into the file where the binary tablebase data
begins.

Within a {\tt <tablebase>} the following elements may
occur in the listed order (deprecated elements and attributes are not
documented):

\subsection{\tt <prune-enable color="white|black" type="concede|discard"/>}

Specifies which kinds of pruning elements will be allowed in this
tablebase and its futurebases.  Both attributes are required.  {\tt
concede} means wins may be conceded to the named color; {\tt discard}
means moves by the named color may be discarded.  At most one {\tt
prune-enable} can be specified for each color.  No {\tt prune-enable}
element is required, however, no {\tt prune} elements are allowed without
one and no futurebases may possess additional {\tt prune-enable}
elements beyond those specified for the current tablebase.


\subsection{\tt <variant name="normal|suicide"/>}

The optional {\tt <variant>} element specifies which version
of the rules of chess apply to this tablebase.

{\bf Default:} {\tt normal}


\subsection{\tt <index type="naive|naive2|simple|compact|no-en-passant|combinadic \hfil\break\hbox{\qquad\qquad\qquad\qquad} |combinadic2|combinadic3|combinadic4|pawngen" \hfil\break\hbox{\qquad} symmetry="1|2|4|8"/>}

The {\tt <index>} element specifies the algorithm that will be used to
compute the index numbers in the tablebase; i.e, the algorithm that
will convert board positions into tablebase offsets and vice versa.
It typically is not specified by the user (but can be).

\begin{description}

\item {\tt naive} uses $2^{6n+1}$ indices to store positions for $n$ pieces.
It assigns a single bit for the side-to-move flag, then assigns 6 bits
to each piece, which is used to encode a number from 0 to 63,
indicating the piece's position on the board.

\item {\tt naive2} Differs from {\tt naive} in its handling of multiple
identical pieces, which it stores as a base and an offset, thus saving
a single bit.  Currently, only pairs of identical pieces are handled;
a fatal error will result if there are more than two identical pieces.

\item {\tt simple} Like {\tt naive}, but only assigns numbers to
squares that are legal for a particular piece.  Slower to compute than
{\tt naive}, but more compact for tablebases with lots of movement
restrictions on the pieces.

\item {\tt compact} A combination of the delta encoding used for
identical pieces in {\tt naive2}, the encoding of restricted pieces used
in {\tt simple}, plus a paired encoding of the kings so they can never be
adjacent.

\item {\tt no-en-passant} An enhancement of {\tt compact} that uses
the paired encoding scheme for pawns restricted to the same file.
Since they can never pass each other, we can encode them as if they
were an identical pair, then assign their colors in the same order
they were originally specified.  En passant significantly complicates
this and can not be handled with this scheme.

\item {\tt combinadic} An enhancement of {\tt naive2} that can encode
  more than two identical, overlapping pieces by using a combinadic
  encoding scheme (see the wikipedia page ``Combinatorial number
  system'').

\item {\tt combinadic2} Like {\tt combinadic}, but later pieces wholly
  contained within the semilegal range of earlier pieces are encoded
  using fewer positions.  Piece order is significant.  No attempt is
  made to reduce the encoding of pawns.

\item {\tt combinadic3} Like {\tt combinadic2}, but pawn encodings are
  also reduced, by reducing its encoding value (with en-passant
  factored in), while using its board position to reduce other pieces..

\item {\tt combinadic4} Like {\tt combinadic3}, but {\it color
  symmetric} tablebases, those invariant under swapping colors,
  are optimised by removing the side-to-move flag, cutting the
  tablebase size in half.  (ex: {\tt kqkq} is color symmetric,
  but {\tt kqkr} is not)

\item {\tt pawngen} Like {\tt combinadic4}, but pawns are handled
  separately by building a table of all possible pawn positions that
  can result from a given initial pawn configuration.

\end{description}

The optional {\tt symmetry} attribute can be used to encode multiple
positions using a single entry, but its utility depends upon the exact
analysis being done.  A tablebase with no pawns and no movement
restrictions can be encoded with 8-way symmetry,
since the board can be rotated about a horizontal, vertical, or
diagonal axis without affecting the behavior of the pieces.  A
tablebase with pawns can utilize at most 2-way
symmetry, since only a reflection about a vertical axis preserves
piece behavior.  A tablebase with restrictions on the positions of the
pieces (say, frozen pawns) can not use any symmetry at all.
Not all symmetries are compatible will all index types; for example,
8-way symmetry can not be used with {\tt naive} or {\tt naive2}
index types.

{\bf Default:} {\tt combinadic4} with automatically selected symmetry,
unless a {\tt pawngen} element is present in the XML, which triggers
{\tt pawngen}

\subsection{Tablebase format}

The next three elements specify the format of the tablebase entries.
At most one of them can be specified.

{\tt <dtm bits={\it integer}/>} specifies a {\it distance to mate}
metric.  Zero is used for draws, -1 is used for positions where the
moving side is checkmated, and 1 is used for positions where the
moving side can capture the opposing king, so an eight bit dtm field
can record mate-in distances up to 126.  If a field size is not
specified, it is selected automatically.

{\tt <dtc bits={\it integer}/>} specifies a {\it distance to
  conversion} metric, which is the number of moves required before
reaching a different tablebase.  Zero is used for draws, -1 is used
for positions where the moving side is checkmated, and 1 is used for
positions where the moving side can capture the opposing king, so an
eight bit dtm field can record distances up to 126.  If a field size
is not specified, it is selected automatically.

{\tt <basic/>} specifies a {\it bitbase} where two bits are used for
each position, and no distance information is stored --- only an
indication of the ultimate outcome (win, lose, or draw).  Such a format
is more compact and requires less time to generate, but requires more
effort to use, since care must be taken to avoid loops when following
winning lines.

{\tt <flag type="white-wins|white-draws"/>} specifies a {\it bitbase}
where only a single bit is used for each position.  {\tt white-draws}
includes both winning and drawing positions for white, so it is
essentially {\tt NOT black-wins}.

{\bf Default:} DTM with automatically selected field size.


\subsection{\tt <piece color="white|black" type="king|queen|rook|bishop|knight|pawn" \hfil\break\hbox{\qquad} location="{\it string}"/>}

Multiple {\tt piece} elements are used to specify the chess pieces
present in the tablebase. {\tt color} and {\tt type} are required and
should be obvious.  The ordering of {\tt piece} elements is
significant in that it directly affects the index algorithm,
but there is no user-visible effect of the ordering.

The optional {\tt location} attribute restricts the board positions
available to this piece.  It should be a list of squares, in algebraic
notation, on which the piece is to be allowed.  A single square
results in a completely frozen piece.  In addition, pawns may use an
additional syntax consisting of a single starting square followed by a
plus sign, indicating that the pawn may move forward as far as
possible.  This can be used, for example, to locate a black pawn on
{\tt "a7+} and a white pawn on {\tt "a2+"}, indicating that both can
move forward, but they can not ``pass'' each other.

\subsection{\tt <pawngen white-pawn-locations="{\it string}" black-pawn-locations="{\it string}" \hfil\break\hbox{\qquad} white-pawns-required="{\it number}" black-pawns-required="{\it number}" \hfil\break\hbox{\qquad} white-queens-required="{\it number}" black-queens-required="{\it number}" \hfil\break\hbox{\qquad} white-captures-allowed="{\it number}" black-captures-allowed="{\it number}"/>}

Pawn configurations can be specified using a {\tt pawngen} element
instead of {\tt piece} elements.  A set of starting pawn locations is
specified for each color using {\tt *-pawn-locations} attributes, and
all possible pawn moves from that initial configuration are
calculated.  Since the number and types of pieces is fixed for each
run of the program, the {\tt *-pawns-required} attributes must be
specified to indicate how many pawns of each color are allowed.
Optionally, the {\tt *-queens-required} attributes can specified to
force consideration of positions where a certain number of pawns have
queened, and the {\tt *-captures-allowed} attributes include
consideration of positions where a certain number of non-pawn pieces
have been captured (pawn captures are already considered).

The auxiliary Perl script {\tt pawngen} can accept control files
without {\tt *-pawns-required} attributes, and generates new,
interlinked control files with the required attributes added.

\subsection{\tt <futurebase filename="{\it string}" colors="invert"/>}

One or more futurebases may be specified with this element.  A {\tt
  filename} must be specified to locate a futurebase, which must be
another tablebase, in either Hoffman, Nalimov, or Syzygy format.  The
path to a Nalimov tablebase is ignored; it must be in the directory
specified on the command line using the {\tt -N} option.

The futurebase must be related to the current tablebase in one of the
following ways:

\begin{description}
\item It has exactly the same piece configuration as the
current tablebase, and corresponds to movement by one of the
restricted pieces, i.e, the current tablebase has a white pawn frozen
on {\tt e4} and the futurebase has a white pawn frozen on {\tt e5}.

\item It has exactly the same piece configuration as
the current tablebase except that a single piece is missing, i.e,
a capture occurred.

\item It has exactly the same piece configuration as the current
tablebase except that a single pawn has been replaced with a knight,
bishop, rook or queen, i.e, a pawn promoted.

\item It has exactly the same piece configuration as the current
tablebase except that a single pawn has been replaced with a knight,
bishop, rook or queen, and a single non-pawn of the opposite color has
been removed, i.e, a pawn captured and promoted in the same move.

\end{description}

The option {\tt colors="invert"} attribute may be specified to indicate
that the piece colors of the futurebase should be inverted as it is
processed.  This precludes the need to calculate, say, a tablebase
with a white queen and a black rook as well as a tablebase with a
black queen and a white rook.  The first may be used (with this
option) as a futurebase to calculate a tablebase with two white rooks
and a black queen.

{\bf Note:} Any futurebase {\tt prune-enable} elements must be a subset of
the current tablebase's {\tt prune-enable} elements.

\subsection{\tt <prune color="white|black" move="{\it string}" type="concede|discard"/>}

Futuremoves not handled by specifying futurebases must be pruned using
one or more of these elements, or an error will result.  The {\tt
  move} is specified using regular expression syntax to match a move
in a subset of standard algebraic notation.  All of the following
strings are examples of legal {\tt move} strings in a {\tt prune}
element: {\tt Pe5}, {\tt P=Q}, {\tt RxQ}, {\tt PxR=Q}.  The following
regular expressions would all match {\tt Kd4}: {\tt Kd?}, {\tt K?4},
{\tt K[a-d]4}, {\tt K*}.  The {\tt type} attribute specifies what
should be done with matching moves: treated as wins for the moving
side ({\tt concede}), or completely ignored ({\tt discard}).  If
multiple {\tt prune} elements match a particular move, it is a warning
if they have the same {\tt type}, a fatal error if their {\tt type}s
differ.

A single {\tt prune} element may be specified with {\tt
move="stalemate"} and {\tt type="concede"}.  In this case, the {\tt
color} attribute indicates to which side stalemates should be conceded
as wins.

{\bf Note:} {\tt prune} elements do not affect moves within a
tablebase.  Specifying a {\tt prune} element that only matches
moves within a tablebase will do {\it nothing}.

{\bf Note:} If a {\tt prune} element is specified for a futuremove
handled by a futurebase, then the futurebase takes precedence.
However, this case is handled by tracking every futuremove in every
position, so it is possible to specify futurebases that handle a
subset of the possible futuremoves, then use {\tt prune} elements to
handle the rest by default.

{\bf Note:} {\tt prune} elements are only allowed if they match a {\tt
prune-enable} element.  If no {\tt prune-enable} elements were
specified, then no {\tt prune} elements will be permitted.

{\bf Note:} Earlier versions of Hoffman allowed a {\tt pawngen-condition}
attribute that is no longer supported.

\subsection{Generation Controls}

These elements are all optional, but if {\tt output} is not specified,
an output filename must be specified on the command line using
the {\tt -o} switch.

\subsubsection{\tt <output filename="{\it string}"/>}

At most a single {\tt output} element should be used to specify where
the finished tablebase should be written.

\vfill\eject

\subsection{\tt <tablebase-statistics> \ldots\quad </tablebase-statistics>}

This element is added by the program and should not be specified in the
input.  It contains statistics relating to the finished tablebase.

\begin{tabular}{c p{12cm}}
Element & \multicolumn{1}{c}{Interpretation} \cr
\hline
{\tt indices} & Total number of entries in the uncompressed tablebase \cr
{\tt PNTM-mated-positions} & Total number of positions in which {\it player not-to-move} is mated; i.e, illegal positions
in which a kind can be immediately captured \cr
{\tt legal-positions} & Total number of legal positions; i.e, total number of entries, minus illegal entries
where two pieces occupy the same space, minus PNTM-mated positions \cr
{\tt stalemate-positions} & Stalemate (not draw by repetition) positions \cr
{\tt white-wins-positions} & Positions from which White can force a win \cr
{\tt black-wins-positions} & Positions from which Black can force a win \cr
{\tt forward-moves} & Total number of forward moves from positions in this tablebase (including futuremoves) \cr
{\tt futuremoves} & Total number of forward moves from positions in this tablebase into futurebases or pruned \cr
{\tt max-dtm} & Largest {\it distance to mate} of all positions in this tablebase \cr
{\tt min-dtm} & Smallest {\it distance to mate} of all positions in this tablebase, i.e, a negative number indicating
the longest forced loss \cr
\end{tabular}

\subsection{\tt <generation-statistics> \ldots\quad </generation-statistics>}

This element is added by the program and should not be specified in the
input.  It contains statistics relating to the program run that generated the tablebase.

\begin{tabular}{c p{12cm}}
Element & \multicolumn{1}{c}{Interpretation} \cr
\hline
{\tt host} & Hostname of system that generated the tablebase \cr
{\tt program} & Name and version of the program that generated the tablebase \cr
{\tt args} & Command line used for the generation run \cr
{\tt start-time} & Time the program run initially started \cr
{\tt completion-time} & Time the program run finally ended \cr
{\tt user-time} & CPU time used by the run in user space \cr
{\tt system-time} & CPU time used by the run in system calls \cr
{\tt real-time} & Wall clock time used by the run \cr
{\tt page-faults} & Number of times the program had to wait for a memory page to be swapped in from disk \cr
{\tt page-reclaims} & Number of times the program reclaimed a page from the free list; this will typically be
program instruction pages \cr
{\tt proptable-writes} & If proptables are in use, the number of proptables written to disk \cr
{\tt proptable-write-time} & If proptables are is use, the total real time required for all proptable writes \cr
{\tt pass} & Per-pass statistics, including {\tt real-time} and {\tt user-time} \cr
\end{tabular}

\vfill\eject

\section{Some Confusing Error Messages}

% This section does not attempt to list all of Hoffman's error messages,
% only those anticipated to cause some confusion.

\subsection{Futurebases can't be less symmetric than the tablebase under construction}

Symmetric tablebases collapse multiple positions into one, so
futurebases must have the same symmetry (at least), or the futurebase
might handle differently two positions that the more symmetric
tablebase treats as one.

\subsection{Doubled pawns must (currently) appear in board order in piece list}

Currently, doubled pawns using ``plus'' locations (ex: {\tt
location="a2+"}) on the same file must have their {\tt piece} elements
listed in the XML in the order that the pawns appear on the board,
counting in algebraic notation from row 1 to row 8.  I mean, row 2 to
row 7.

\subsection{Piece restrictions not allowed with symmetric indices (yet)}

You can't specify an {\tt index symmetry} attribute and also specify
{\tt piece location} attributes, even if the restrictions on the piece
locations might be compatible with the requested symmetry.

\subsection{Non-identical overlapping piece restrictions not allowed with this index type}

For the {\tt naive}, {\tt naive2}, and {\tt simple} index types, you
can't specify two identical pieces with different {\tt location}
restrictions unless those restrictions are completely distinct.  For
example, you can't have a free white rook and another white rook
restricted to the a-file.  If you think about it, this situation would
allow the rooks to ``trade places'' --- both could move to the a-file
and then either one could move off.  The simpler index types can't
handle this situation.  You could, however, have a white rook
restricted to the a-file and another restricted to the d-file (or use
a more sophisticated index type, like {\tt compact}).

\subsection{Futurebase doesn't match prune-enables!}

Remember that futurebase {\tt prune-enable} elements must be a
subset of the current tablebase's {\tt prune-enable}s.


\subsection{No futurebase or pruning for ... \hfil\break Futuremoves not handled ...}

If one or more futuremoves are not handled by specifying either a
futurebase or a prune statement, then a fatal error will result either
immediately or after the initialization pass.  To aid in diagnosis,
the error message includes the FEN of the offending position.

\subsection{pawngen doesn't support output elements in generation-controls}

The {\tt pawngen} script automatically generates all of its filenames.
Remove the output element.  After generation, all of the resulting
{\tt htb} files can be loaded together into Hoffman's probe mode.


\end{document}
