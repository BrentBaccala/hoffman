\documentclass[11pt]{article}

\usepackage{vmargin}
\setmarginsrb{0.5in}{0.5in}{0.5in}{0.5in}{20pt}{20pt}{20pt}{20pt}

\title{The Hoffman Reference Guide}
\author{Brent Baccala}

\usepackage{times}
\pagestyle{empty}

% Get rid of page numbers
\def\thepage{}

\begin{document}

\maketitle

\parindent 0pt
\parskip 12pt

Hoffman is a program to solve chess endgames using retrograde
analysis, which is much different from conventional computer chess
programs.  Retrograde analysis is only useful in the endgame, runs
very slowly, and produces enormous amounts of data.  Its great
advantage lies in its ability to completely solve the endgame.  In a
very real sense, a retrograde engine has no ``move horizon'' like a
conventional chess engine.  It sees everything.  For those not up on
Americana, the program is named after Trevor Hoffman, an All Star
baseball pitcher who specializes in ``closing'' games.  It was written
specifically for The World vs. Arno Nickel game.

Hoffman uses XML extensively both for configuring its operation and
for labeling the resulting tablebases.  In fact, a completed Hoffman
tablebase (typically in an {\tt .htb} file) is just a {\tt gzip}-ed
file that contains an XML prefix followed by binary tablebase data (in
a format specified by the XML).  Basically, to operate Hoffman, you
write an XML file that specifies the analysis you want done, then feed
it to the program.  As output, it produces a modified version of the
XML input that includes binary tablebase data appended at the end.

\section{Parallel Processing with Hoffman}

A Hoffman analysis can be quite compute-intensive.  The program can be
compiled to use POSIX threads (if available), with the number of
threads specified at run-time using the {\tt -p} option.  The program
is also designed to use multiple computers in parallel, all working
simultaneously on an analysis.  This is accomplished by breaking the
analysis up into smaller pieces, each with its own XML configuration
file.  The primary support provided by the program is the ability to
use URLs instead of filenames to reference tablebases, allowing
tablebases to be stored on a server accessable over a network.  A
basic Perl CGI script ({\tt hoffman.cgi}) is provided which, when
installed on a web server and supplied with a directory full of
Hoffman XML files, will hand them out in the proper order to Hoffman
clients for processing.  Unfortunately, the process of creating those
multiple interlinked XML files remains largely unautomated at this
time.

\section{Propagation tables}

A Hoffman analysis can also be quite space-intensive.  Since its
memory utilization pattern is basically random, Hoffman will begin to
swap dramatically and suffer a disastrous drop in performance once its
working set size exceeds the machine's available memory.  To alleviate
this, the program can be operated in a mode where it fills a series of
{\it propagation tables}, writing each one out to disk when full, then
reads them back in sequentially during the next pass.  Although less
efficient than when the working set can be contained in memory,
propagation tables allow the program to build tablebases of
essentially unlimited size with no swapping and reasonable CPU
utilization.  This mode is activated at run-time by specifying the
size of the propagation tables (in MB) with the {\tt -P} switch.

{\bf Note:} If you're using proptables, performance will be abysmal
unless you specify a {\tt modulus} attribute to the {\tt index}
element.


\vfil\eject
\section{XML Syntax}

The root XML element in a Hoffman tablebase is always {\tt
<tablebase>}.  Its only attribute ({\tt offset}) is added by the
program, should not be supplied by the user, and indicates a
hexadecimal byte-offset into the file where the binary tablebase data
begins.

Within a {\tt <tablebase>} the following elements may
occur in the listed order (deprecated elements and attributes are not
documented):

\subsection{\tt <prune-enable color="white|black" type="concede|discard"/>}

Specifies which kinds of pruning elements will be allowed in this
tablebase and its futurebases.  Both attributes are required.  {\tt
concede} means wins may be conceded to the named color; {\tt discard}
means moves by the named color may be discarded.  At most one {\tt
prune-enable} can be specified for each color.  No {\tt prune-enable}
element is required, however, no {\tt prune} elements are allowed without
one and no futurebases may possess additional {\tt prune-enable}
elements beyond those specified for the current tablebase.


\subsection{\tt <index type="naive|naive2|simple|compact|no-en-passant" \hfil\break\hbox{\qquad} symmetry="1|2|2-way|4|8|8-way" modulus="auto|{\it integer}"/>}

The {\tt <index>} element specifies the algorithm that will be used to
compute the index numbers in the tablebase; i.e, the algorithm that
will convert board positions into tablebase offsets and vice versa.

\begin{description}

\item {\tt naive} uses $2^{6n+1}$ indices to store positions for $n$ pieces.
It assigns a single bit for the side-to-move flag, then assigns 6 bits
to each piece, which is used to encode a number from 0 to 63,
indicating the piece's position on the board.

\item {\tt naive2} Differs from {\tt naive} in its handling of multiple
identical pieces, which it stores as a base and an offset, thus saving
a single bit.  Currently, only pairs of identical pieces are handled;
a fatal error will result if there are more than two identical pieces.

\item {\tt simple} Like {\tt naive}, but only assigns numbers to
squares that are legal for a particular piece.  Slower to compute than
{\tt naive}, but more compact for tablebases with lots of movement
restrictions on the pieces.

\item {\tt compact} A combination of the delta encoding used for
identical pieces in {\tt naive2}, the encoding of restricted pieces used
in {\tt simple}, plus a paired encoding of the kings so they can never be
adjacent.

\item {\tt no-en-passant} An enhancement of {\tt compact} that uses
the paired encoding scheme for pawns restricted to the same file.
Since they can never pass each other, we can encode them as if they
were an identical pair, then assign their colors in the same order
they were originally specified.  En passant significantly complicates
this and can not be handled with this scheme.

\end{description}

The optional {\tt symmetry} attribute can be used to encode multiple
positions using a single entry, but its utility depends upon the exact
analysis being done.  A tablebase with no pawns and no movement
restrictions can be encoded with {\tt 8-way} (alias {\tt 8}) symmetry,
since the board can be rotated about a horizontal, vertical, or
diagonal axis without affecting the behavior of the pieces.  A
tablebase with pawns can utilize at most {\tt 2-way} (alias {\tt 2})
symmetry, since only a reflection about a vertical axis preserves
piece behavior.  A tablebase with restrictions on the positions of the
pieces (say, frozen pawns) can not use any symmetry at all ({\tt 1}).
Not all symmetries are compatible will all index types; for example,
8-way symmetry can not be used with {\tt naive} or {\tt naive2}
index types.
{\bf Default:} {\it no symmetry}

The optional {\tt modulus} attribute, which if specified should be
either {\tt auto} or a prime number (the program complains if it's
composite), indicates that the computed index should be inverted in a
finite field (modulo the specified number) to obtain the actual index.
While time consuming, this step has the effect of shuffling the
indices in a pseudo-random fashion, and should be used if proptables
are in use in order to optimize the operation of the library sort.
This also produces larger tablebases, since a pseudo-random
distribution of mating positions impedes the operation of the {\tt
gzip} compression algorithm.  {\tt auto} simply rounds the highest
index up to the next prime number; there really is no reason anymore
to specify a specific prime.
{\bf Default:} {\it no inversion}

\subsection{\tt <format> \ldots\quad </format>}

This element specifies the format of the tablebase entries.  It has no
attributes, and must contain exactly one of the following two
elements:

{\tt <dtm bits="8|16"/>} specifies a {\it distance to mate} metric
occupying either one or two bytes.  Zero is used for draws, -1 is used
for positions where the moving side is checkmated, and 1 is used for
positions where the moving side can capture the opposing king, so a
one byte dtm can record mate-in distances up to 126.  A two byte dtm
has no such (practical) limitation.

{\tt <flag type="white-wins|white-draws"/>} specifies a {\it bitbase}
where only a single bit is used for each position, and no distance
information is stored --- only an indicator of the ultimate outcome.
Such a format is more compact and requires less time to generate, but
requires more effort to use, since care must be taken to avoid loops
when following winning lines.

{\bf Note:} Bitbases have been rarely used and remain poorly tested.

{\bf Default:} 8-bit DTM.


\subsection{\tt <piece color="white|black" type="king|queen|rook|bishop|knight|pawn" \hfil\break\hbox{\qquad} location="{\it string}"/>}

Multiple {\tt piece} elements are used to specify the chess pieces
present in the tablebase. {\tt color} and {\tt type} are required and
should be obvious.  The ordering of {\tt piece} elements is
significant in that it directly affects the index algorithm,
but there is no user-visible effect of the ordering.

The optional {\tt location} attribute restricts the board positions
available to this piece.  It should be a list of squares, in algebraic
notation, on which the piece is to be allowed.  A single square
results in a completely frozen piece.  In addition, pawns may use an
additional syntax consisting of a single starting square followed by a
plus sign, indicating that the pawn may move forward as far as
possible.  This can be used, for example, to locate a black pawn on
{\tt "a7+} and a white pawn on {\tt "a2+"}, indicating that both can
move forward, but they can not ``pass'' each other.


\subsection{\tt <futurebase filename="{\it string}" url="{\it string}" colors="invert"/>}

One or more futurebases may be specified with this element.  Either a
{\tt filename} or a {\tt url} may be specified (not both) to locate a
futurebase, which must be another Hoffman tablebase.  It must be
related to the current tablebase in one of the following ways:

\begin{description}
\item It has exactly the same piece configuration as the
current tablebase, and corresponds to movement by one of the
restricted pieces, i.e, the current tablebase has a white pawn frozen
on {\tt e4} and the futurebase has a white pawn frozen on {\tt e5}.

\item It has exactly the same piece configuration as
the current tablebase except that a single piece is missing, i.e,
a capture occurred.

\item It has exactly the same piece configuration as the current
tablebase except that a single pawn has been replaced with a knight,
bishop, rook or queen, i.e, a pawn promoted.

\item It has exactly the same piece configuration as the current
tablebase except that a single pawn has been replaced with a knight,
bishop, rook or queen, and a single non-pawn of the opposite color has
been removed, i.e, a pawn captured and promoted in the same move.

\end{description}

The option {\tt colors="invert"} attribute may be specified to indicate
that the piece colors of the futurebase should be inverted as it is
processed.  This precludes the need to calculate, say, a tablebase
with a white queen and a black rook as well as a tablebase with a
black queen and a white rook.  The first may be used (with this
option) as a futurebase to calculate a tablebase with two white rooks
and a black queen.

{\bf Note:} Any futurebase {\tt prune-enable} elements must be a subset of
the current tablebase's {\tt prune-enable} elements.

\subsection{\tt <prune color="white|black" move="{\it string}" type="concede|discard"/>}

Futuremoves not handled by specifying futurebases must be pruned using
one or more of these elements, or an error will result.  All three
attributes are required.  The {\tt move} is specified using regular
expression syntax to match a move in a subset of standard algebraic
notation.  All of the following strings are examples of legal {\tt
move} strings in a {\tt prune} element: {\tt Pe5}, {\tt P=Q}, {\tt
RxQ}, {\tt PxR=Q}.  The following regular expressions would all match
{\tt Kd4}: {\tt Kd?}, {\tt K?4}, {\tt K[a-d]4}, {\tt K*}.
The {\tt type} attribute specifies what should be done with matching
moves: treated as wins for the moving side ({\tt concede}), or
completely ignored ({\tt discard}).  If multiple {\tt prune}
elements match a particular move, it is a warning if they have the
same {\tt type}, a fatal error if their {\tt type}s differ.

A single {\tt prune} element may be specified with {\tt
move="stalemate"} and {\tt type="concede"}.  In this case, the {\tt
color} attribute indicates to which side stalemates should be conceded
as wins.  My main idea in implementing this was to speed up processing
by allowing promotion futurebases with only queens to substitute for
promotions into rooks and bishops, since the only logical reason to
underpromote is to avoid a stalemate.  To this end, a futurebase with
an extra queen and stalemates pruned will be treated as handling all
three promotions (queen, rook, and bishop).  This feature has not been
tested, and I believe there are bugs in the code (I'm not sure if the
color sense is right, and I think you can ``lose'' these statements
through a series of linked futurebases).

{\bf Note:} {\tt prune} elements are only allowed if they match a {\tt
prune-enable} element.  If no {\tt prune-enable} elements were
specified, then no {\tt prune} elements will be permitted.

\subsection{\tt <generation-controls> \ldots\qquad </generation-controls>}

This optional element has no attributes and contains one or more
of the following sub-elements, in no particular order:

\subsubsection{\tt <output filename="{\it string}" url="{\it string}"/>}

At most a single {\tt output} element should be used, with either a
{\tt filename} or a {\tt url} (but not both), to specify where
the finished tablebase should be written.

{\bf Note:} The only {\tt url} scheme that has been tested for this
element is {\tt ftp}.

\subsubsection{\tt <completion-report url="{\it string}"/> \hfil\break <error-report url="{\it string}"/>}

Optionally, the tablebase's XML prefix (without the tablebase data) can
be written to a URL upon either a successful or error termination of
the program.  This capability (along with the ability to be read and
write tablebase URLs) is intended to aid the construction of Hoffman
tablebases using a distributed cluster.

{\bf Note:} In the event of an error termination, every attempt will
be made to add {\tt <error>} elements to the XML indicating the
cause of the problem.

\subsubsection{\tt <entries-format> \ldots\quad </entries-format>}

This optional entity controls the internal format used to store a
tablebase during its construction.  It contains a number of entities,
each corresponding to a structure field, all of which admit at least
the attributes {\tt bits} and {\tt offset}, which specify,
respectively, the number of bits occupied by the field and the field's
offset relative to the beginning of a tablebase entry.  {\tt offset}
is optional and, if not specified, will be computed using an algorithm
to assign fields to empty slots, though if {\tt offset} is specified
in one element, it must be specified in all of them. {\tt bits} is
usually required, except for single-bit-only fields.  The total number
of bits required must be a power-of-two byte boundary, i.e, 8, 16, 32,
64.  The entities are:

\begin{description}

\item[{\tt dtm}] Exactly as specified in the {\tt format}.

\item[{\tt flag}] Single-bit-only field.  Exactly as specified in the
{\tt format}.  A {\tt type} attribute (as in {\tt format}) must be
specified.  Should be specified instead of a {\tt dtm} field if a
bitbase is being constructed.

\item[{\tt movecnt}] Used to count the number of possible moves from a
given position.  Has four reserved values, and must be able to count
down to zero, so an $n$-bit {\tt movecnt} allows positions with up to
$2^n-5$ moves.  If 8-way symmetry is in use, then many positions
require their moves to be counted twice, effectively halving that
number.  If positions exist with too many possible moves to fit into
{\tt movecnt}, a fatal error will result during tablebase
initialization.

\item[{\tt locking-bit}] Single-bit-only field.  Optional, but
recommended if running multi-threaded.  Allows individual tablebase
entries to be locked.  Without it, a global lock must be used to
regulate access to the entries table, which can adversely affect
performance, but probably not as much as doubling the size of the
entries table if that is the only way to create a free bit.

\end{description}

\begin{tabular}{@{} l l}
{\bf Default:} & {\tt <entries-format>} \cr
& \qquad {\tt <dtm bits="8" offset="0"/>} \cr
& \qquad {\tt <movecnt bits="7" offset="9"/>} \cr
& \qquad {\tt <locking-bit offset="8"/>} \cr
& {\tt </entries-format>} \cr
\end{tabular}

{\bf Note:} Changing {\tt entries-format} requires the program to be
recompiled with a different {\tt formats.h} file, which the
program will print as it terminates with a fatal error.

{\bf Example:} A good example of how to use this entity is found in
the standard {\tt kppkp} tablebase, which has a maximum distance to
mate of 128, and therefore requires a 9 bit {\tt dtm} field
(unfortunately, there's no way to know this in advance).  This can be
achieved within a 2-byte {\tt entries-format} by stealing a bit from
the {\tt movecnt} field, leaving it with 6 bits for a possible $2^6 -
5 = 59$ movements.  Since a king can have no more than 8 movements,
and a pawn can have no more than 12 (on the seventh rank it has two
possible captures and a forward movement, times four possible pieces
it can promote into), a king and two pawns can have no more than 32
movements (less, actually, since no capture-promotions are possible
with only a pawn and a king on the opposing side), so a 6-bit {\tt
movecnt} works fine.  On the other hand, this alternate format would
not work on the {\tt kqqkq} tablebase, since a queen can have up to 28
movements, so a king and two queens can have around 74 movements, which
gets doubled to 148 since we use 8-way symmetry.


\subsubsection{\tt <proptable-format> \ldots\quad </proptable-format>}

This optional entity controls the internal format used to store
proptables, and is structured like {\tt entries-format}.
The possible entries here are:

\begin{description}

\item[{\tt dtm}] Exactly as specified in the {\tt format}.

\item[{\tt PTM-wins-flag}] Single-bit-only field.  Similar to the {\tt
flag} field in {\tt entries-format}, but has a slightly different
interpretation (indicates if {\it player to move} wins, not white).
No {\tt type} attribute.  Should be specified instead of a {\tt dtm}
field if a bitbase is being constructed.

\item[{\tt movecnt}] Similar to the same field in {\tt
entries-format}, but has no reserved values and doesn't have to hold a
complete move count (only the number of moves being propagated).
Currently has {\bf no check} for overflow!!

\item[{\tt index-field}] Holds the index number being propagated.
See the {\tt index} section for more information about how
index numbers are computed.
Currently has {\bf no check} for overflow!!

\item[{\tt futurevector}] Only used during the first back-propagation
pass, to track which futuremoves have been handled as futurebases are
back-propagated.  Must be large enough to hold a single bit for each
possible futuremove from a position.  Hoffman will die (early) with a
fatal error if this field is not large enough.

\end{description}

\begin{tabular}{@{} l l}
{\bf Default:} & {\tt <proptable-format>} \cr
& \qquad {\tt    <index bits="32" offset="0"/>} \cr
& \qquad {\tt    <dtm bits="16" offset="32"/>} \cr
& \qquad {\tt    <movecnt bits="8" offset="56"/>} \cr
& \qquad {\tt    <futurevector bits="64" offset="64"/>} \cr
& {\tt </proptable-format>} \cr
\end{tabular}

{\bf Note:} Changing {\tt proptable-format} requires the program to be
recompiled with a different {\tt formats.h} file, which the
program will print as it terminates with a fatal error.

\subsection{\tt <tablebase-statistics> \ldots\quad </tablebase-statistics>}

This entity is added by the program and should not be specified in the
input.  It contains statistics relating to the finished tablebase.

\begin{tabular}{c p{12cm}}
Entity & \multicolumn{1}{c}{Interpretation} \cr
\hline
{\tt indices} & Total number of entries in the uncompressed tablebase \cr
{\tt PNTM-mated-positions} & Total number of positions in which {\it player not-to-move} is mated; i.e, illegal positions
in which a kind can be immediately captured \cr
{\tt legal-positions} & Total number of legal positions; i.e, total number of entries, minus illegal entries
where two pieces occupy the same space, minus PNTM-mated positions \cr
{\tt stalemate-positions} & Stalemate (not draw by repetition) positions \cr
{\tt white-wins-positions} & Positions from which White can force a win \cr
{\tt black-wins-positions} & Positions from which Black can force a win \cr
{\tt forward-moves} & Total number of forward moves from positions in this tablebase (including futuremoves) \cr
{\tt futuremoves} & Total number of forward moves from positions in this tablebase into futurebases or pruned \cr
{\tt max-dtm} & Largest {\it distance to mate} of all positions in this tablebase \cr
{\tt min-dtm} & Smallest {\it distance to mate} of all positions in this tablebase, i.e, a negative number indicating
the longest forced loss \cr
\end{tabular}

\subsection{\tt <generation-statistics> \ldots\quad </generation-statistics>}

This entity is added by the program and should not be specified in the
input.  It contains statistics relating to the program run that generated the tablebase.

\begin{tabular}{c p{12cm}}
Entity & \multicolumn{1}{c}{Interpretation} \cr
\hline
{\tt host} & Hostname of system that generated the tablebase \cr
{\tt program} & Name and version of the program that generated the tablebase \cr
{\tt args} & Command line used for the generation run \cr
{\tt start-time} & Time the program run initially started \cr
{\tt completion-time} & Time the program run finally ended \cr
{\tt user-time} & CPU time used by the run in user space \cr
{\tt system-time} & CPU time used by the run in system calls \cr
{\tt real-time} & Wall clock time used by the run \cr
{\tt page-faults} & Number of times the program had to wait for a memory page to be swapped in from disk \cr
{\tt page-reclaims} & Number of times the program reclaimed a page from the free list; this will typically be
program instruction pages \cr
{\tt proptable-writes} & If proptables are in use, the number of proptables written to disk \cr
{\tt proptable-write-time} & If proptables are is use, the total real time required for all proptable writes \cr
{\tt pass} & Per-pass statistics, including {\tt real-time} and {\tt user-time} \cr
\end{tabular}

\section{Error Messages}

This section does not attempt to list all of Hoffman's error messages,
only those anticipated to cause some confusion.

\subsection{Doubled pawns must (currently) appear in board order in piece list}

Currently, doubled pawns using ``plus'' locations (ex: {\tt
location="a2+"}) on the same file must have their {\tt piece} elements
listed in the XML in the order that the pawns appear on the board,
counting in algebraic notation from row 1 to row 8.  I mean, row 2 to
row 7.

\subsection{Piece restrictions not allowed with symmetric indices (yet)}

You can't specify an {\tt index symmetry} attribute and also specify
{\tt piece location} attributes, even if the restrictions on the piece
locations might be compatible with the requested symmetry.

\subsection{Non-identical overlapping piece restrictions not allowed with this index type}

For the {\tt naive}, {\tt naive2}, and {\tt simple} index types, you
can't specify two identical pieces with different {\tt location}
restrictions unless those restrictions are completely distinct.  For
example, you can't have a free white rook and another white rook
restricted to the a-file.  If you think about it, this situation would
allow the rooks to ``trade places'' --- both could move to the a-file
and then either one could move off.  The simpler index types can't
handle this situation.  You could, however, have a white rook
restricted to the a-file and another restricted to the d-file (or use
a more sophisticated index type, like {\tt compact}).

\subsection{Attempting to initialize position with a movecnt that won't fit in field!}

The {\tt movecnt} field specified in {\tt generation-controls} wasn't big enough,
and note that the default value might not be big enough!

\subsection{Default/specified proptable/entries format incompatible with compiled-in format}

Changing the formats in {\tt generation-controls} requires recompiling
the program with a different {\tt "formats.h"} file.

\subsection{Futurebase doesn't match prune-enables!}

Remember that futurebase {\tt prune-enable} elements must be a
subset of the current tablebase's {\tt prune-enable}s.

\end{document}
