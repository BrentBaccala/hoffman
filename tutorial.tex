\documentclass[11pt]{article}

\usepackage{skak}

\usepackage{vmargin}
\setmarginsrb{0.5in}{0.75in}{0.5in}{0.5in}{20pt}{20pt}{20pt}{20pt}

\title{The Hoffman Tutorial}
\author{Brent Baccala}

\usepackage{times}
\pagestyle{empty}

% Get rid of section numbers
\def\thesection{}

% Get rid of page numbers
\def\thepage{}

\begin{document}

\maketitle

\parindent 0pt
\parskip 12pt

Hoffman is a program to solve chess endgames using retrograde
analysis.
A retrograde analysis program is much different from a conventional
computer chess programs.  Retrograde analysis is only useful in the
endgame, runs very slowly, and produces enormous amounts of data.  Its
great advantage lies in its ability to completely solve the endgame.
In a very real sense, a retrograde engine has no ``move horizon'' like a
conventional chess engine.  It sees everything.
For those not up on Americana, the program is named after
Trevor Hoffman, an All Star baseball pitcher who specializes in
``closing'' games.  It was written specifically for The World vs. Arno
Nickel game.

The first thing to understand is that Hoffman uses XML
control files to govern its operation.
Hoffman comes with an {\tt xml} directory containing a number of
these control files.  I'll illustrate how Hoffman works using these
examples.
For example, Figure 1 shows about the simplest possible XML control file, for
the king vs. king endgame, contained in the file {\tt xml/kk.xml}.


\begin{figure}
\hrule\ 
{\small\begin{verbatim}
<?xml version="1.0"?>
<!DOCTYPE tablebase SYSTEM "http://www.freesoft.org/software/hoffman/tablebase.dtd">
<!-- Created by genctlfile.pl -->

<tablebase>
   <piece color="white" type="king"/>
   <piece color="black" type="king"/>
   <output filename="kk.htb"/>
</tablebase>
\end{verbatim}}
\hrule
\caption{\tt kk.xml}
\end{figure}

What does this all mean?  Well, the first two lines identify this as a
Hoffman XML tablebase file; copy them verbatim at the top of any new
Hoffman control file.  The third line illustrates XML's comment
format; anything between {\tt <!--} and {\tt -->} is a comment, in
this case the identity of the Perl script that created this file. The
bulk of the file is contained between the two lines {\tt <tablebase>}
and {\tt </tablebase>}.  {\tt tablebase} is a simple example of an XML
{\it element}, which come in basically two flavors: the simple kind
and the complex kind.  The simple kind, like, say, {\tt piece}, have
only a single tag that ends with a slash.  The complex kind have both
a beginning and an ending tag.  The beginning tag does not have a
slash, and the ending tag is identical to the beginning tag except
that it starts with a slash (like {\tt tablebase}).  Complex elements
allow you to put things between the starting and ending tags.  Follow
my lead; it'll start to make sense soon.

%The {\tt index} element controls how Hoffman numbers the positions in
%the tablebase.  I'm using the ``compact'' index type, which is a
%pretty good choice right now for just about everything, and I've also
%specified a ``symmetry'' to indicate that I only want to store
%positions where the white king is in a little triangular eighth of the
%board.  Why?  Because you can rotate and reflect the board around to
%put the white king anywhere you want him.  Think about it.  Symmetry
%doesn't work with all tablebases.  If there were pawns involved, for
%example, you can't just rotate the board 90 degrees.  But since all we
%have are kings, symmetry is OK, and it makes the resulting tablebase a
%lot smaller.

%Now we have a {\tt format} line, indicating that we want an output
%format with one field --- an eight bit {\it distance to mate} (DTM)
%--- for each position.  In other words, the classic ``mate in N'' or
%``mated in N'', so long as N can fit into eight bits (so it has to be
%``mate in 126'' or less, since DTM is a signed field).  Eight bit DTM
%is big enough for every three, four, or five piece tablebase except
%kppkp, so just copy this line verbatim, too.

The {\tt piece} elements are probably self-explanatory.  Notice that I
 didn't specify where on the board the kings were.  That's because
 we're going to compute results for every chess position possible with
 these two pieces.  Their movements are completely unrestricted ---
 they can be anywhere on the board.  Later we'll see more complex
 piece elements that restrict where on the board the pieces can be.

Finally, we come to {\tt output}, which tells the program where to put
 the output tablebase, in this case, into a file called {\tt kk.htb}.
  You can rename the resulting tablebase file as you wish; the
 tablebase isn't tied to a particular filename.

Now you put all this into a file called something like {\tt kk.xml},
(or just copy it from the {\tt xml} directory) and run Hoffman
like this:


\begin{verbatim}
C> hoffman -g kk.xml
Initializing tablebase
Checking futuremoves...
All futuremoves handled under move restrictions
Intra-table propagating
Pass 0 complete; 840 positions processed
Pass 1 complete; 0 positions processed
C>
\end{verbatim}

There isn't much to see, of course.  The ``{\tt -g}'' option
meant ``generate''.  King vs king is nothing more than figuring out
the difference between illegal positions and draws.  But this
information is important, because it's needed to back propagate from
the (slightly) more complex three piece endgames, like the king and
queen endgame whose XML configuration is shown in Figure 2.

\begin{figure}
\hrule\ 
{\small\begin{verbatim}
<?xml version="1.0"?>
<!DOCTYPE tablebase SYSTEM "http://www.freesoft.org/software/hoffman/tablebase.dtd">
<!-- Created by genctlfile.pl -->

<tablebase>
   <piece color="white" type="king"/>
   <piece color="black" type="king"/>
   <piece color="white" type="queen"/>
   <futurebase filename="kk.htb"/>
   <output filename ="kqk.htb"/>
</tablebase>
\end{verbatim}}
\hrule
\caption{\tt kqk.xml}
\end{figure}

Notice the new ``futurebase'' line.  This tells Hoffman where to get
the information about what happens when the black king can capture the
white queen, because a tablebase contains information for {\it
exactly} the piece configuration it is set up for --- nothing
more, nothing less.

So put all this in a file called {\tt kqk.xml}, make sure the {\tt
kk.htb} file from the first run is present, and run Hoffman again:

\begin{verbatim}
C> ./hoffman -g -o kqk.htb kqk.xml
Initializing tablebase
Back propagating from 'kk.htb'
Checking futuremoves...
All futuremoves handled under move restrictions
Intra-table propagating
Pass 0 complete; 131516 positions processed
Pass 1 complete; 364 positions processed
Pass 2 complete; 2448 positions processed
Pass 3 complete; 1352 positions processed
Pass 4 complete; 5012 positions processed
Pass 5 complete; 2956 positions processed
Pass 6 complete; 9064 positions processed
Pass 7 complete; 7480 positions processed
Pass 8 complete; 19964 positions processed
Pass 9 complete; 14144 positions processed
Pass 10 complete; 26164 positions processed
Pass 11 complete; 25484 positions processed
Pass 12 complete; 32064 positions processed
Pass 13 complete; 39908 positions processed
Pass 14 complete; 32104 positions processed
Pass 15 complete; 54052 positions processed
Pass 16 complete; 15000 positions processed
Pass 17 complete; 43800 positions processed
Pass 18 complete; 2680 positions processed
Pass 19 complete; 11300 positions processed
Pass 20 complete; 8 positions processed
Pass 21 complete; 56 positions processed
Pass 22 complete; 0 positions processed
C>
\end{verbatim}

See, it's a little more interesting this time, right?

Now is a good time to introduce the ``{\tt -i}'' (information) option.
Once you've got a bunch of {\tt .htb} files sitting around, and you
can't remember which XML control file was used to generate which
tablebase, there's no need to panic.  Everything from the original XML
configuration is saved into the resulting tablebase, along with a
bunch more information, and all of it can be retrieved from the
tablebase using {\tt -i}:

{\small\begin{verbatim}
C> hoffman -i kqk.htb
Hoffman $Revision: 1.15 $ $Locker: baccala $
0 piece Nalimov tablebases found
<?xml version="1.0"?>
<!DOCTYPE tablebase SYSTEM "http://www.freesoft.org/software/hoffman/tablebase.dtd">
<tablebase offset="0x0fe4">
   <index type="compact" symmetry="8-way"/>
   <format><dtm bits="8"/></format>
   <piece color="white" type="king"/>
   <piece color="black" type="king"/>
   <piece color="white" type="queen"/>
   <futurebase filename="kk.htb"/>
   <tablebase-statistics>
      <indices>59136</indices>
      <PNTM-mated-positions>10152</PNTM-mated-positions>
      <legal-positions>47136</legal-positions>
      <stalemate-positions>115</stalemate-positions>
      <white-wins-positions>44183</white-wins-positions>
      <black-wins-positions>0</black-wins-positions>
      <forward-moves>686465</forward-moves>
      <futuremoves>2838</futuremoves>
      <max-dtm>11</max-dtm>
      <min-dtm>-11</min-dtm>
   </tablebase-statistics>
   <generation-statistics>
      <host>debian.freesoft.org</host>
      <program>Hoffman $Revision: 1.15 $ $Locker: baccala $</program>
      <args>./hoffman -g -o kqk.htb kqk.xml </args>
      <start-time>Sat Dec 16 02:11:28 2006 EST</start-time>
      <completion-time>Sat Dec 16 02:11:36 2006 EST</completion-time>
      <user-time>2.782s</user-time>
      <system-time>0.018s</system-time>
      <real-time>8.052s</real-time>

      ... about 60 more lines deleted ...

   </generation-statistics>
</tablebase>
\end{verbatim}}

We see the configuration information from the input XML file, plus a
 lot more information added by the program.  We see the encoding
 scheme used to number the board positions (the {\tt index} element),
 the type of information stored for each position (the {\tt dtm}
 element -- Distance To Mate), as well as two entire new sections ---
 {\tt tablebase-statistics} and {\tt generation-statistics}.  The
 first reports various interesting information the program determined
 about the tablebase, such as how many total indices there are, how
 many correspond to legal chess positions, how many white mates, or
 black mates, or stalemates there are, etc.  The second reports
 information about the actual generation of this tablebase, like when
 it occured, which version of the program was used, and which computer
 actually computed it.

OK, so what's next?  After {\tt kk.xml} and {\tt kqk.xml}, then you
can easily understand {\tt krk.xml}, {\tt kbk.xml} and {\tt knk.xml}.
Once all five of these are processed, you're now ready to build {\tt
kpk.xml} (Figure 3).

\begin{figure}
\hrule\ 
{\small\begin{verbatim}
<?xml version="1.0"?>
<!DOCTYPE tablebase SYSTEM "http://www.freesoft.org/software/hoffman/tablebase.dtd">
<!-- Created by genctlfile.pl -->

<tablebase>
   <piece color="white" type="king"/>
   <piece color="black" type="king"/>
   <piece color="white" type="pawn"/>
   <futurebase filename="kk.htb"/>
   <futurebase filename="kqk.htb"/>
   <futurebase filename="krk.htb"/>
   <futurebase filename="kbk.htb"/>
   <futurebase filename="knk.htb"/>
   <output filename ="kpk.htb"/>
</tablebase>
\end{verbatim}}
\hrule
\caption{\tt kpk.xml}
\end{figure}

Notice we've added a new type of {\tt futurebase} --- pawn promotion.
Hoffman has to know what happens after that pawn transforms into a
queen, rook, bishop, or knight to be able to understand what happens
to the pawn!

It's starting to get more complex, right?  So how do I know there
isn't a bug in all of this complexity?  Well, my most important
blunder check is to verify the program's operation against the Nalimov
tablebases.  You can do this, too.  If you download the appropriate
Nalimov tablebases from the Internet (in this case, the two KPK
files), you can verify that Hoffman's results are identical to
Nalimov's using the ``{\tt -v}'' (verify) and ``{\tt -n} {\it
directory}'' (location of Nalimov files) options, like this:

\begin{verbatim}
C> ./hoffman -v -n Nalimov/ kpk.htb
Hoffman $Revision: 1.15 $ $Locker: baccala $
5 piece Nalimov tablebases found
Loading 'kpk.htb'
Verifying tablebase against Nalimov
C>
\end{verbatim}

There were no complaints, so that means everything verified OK.

So now you've got all of the three piece tablebases.  Ready to try a
four piece?  Figure 4 is {\tt kqkq.xml}.

\begin{figure}
\hrule\ 
{\small\begin{verbatim}
<?xml version="1.0"?>
<!DOCTYPE tablebase SYSTEM "http://www.freesoft.org/software/hoffman/tablebase.dtd">
<!-- Created by genctlfile.pl -->

<tablebase>
   <piece color="white" type="king"/>
   <piece color="black" type="king"/>
   <piece color="white" type="queen"/>
   <piece color="black" type="queen"/>
   <futurebase filename="kqk.htb" colors="invert"/>
   <futurebase filename="kqk.htb"/>
   <output filename ="kqkq.htb"/>
</tablebase>
\end{verbatim}}
\hrule
\caption{\tt kqkq.xml}
\end{figure}

Notice several things.  First, we no longer specify {\tt kk.htb} as a
futurebase.  We're only interested in the single captures that lead
out of the tablebase we're building.  {\tt kk.htb} has already been
used to build {\tt kqk.htb}, so its data is in there.  Two queens
can't be taken on a single move, so all we need to worry about is what
happens if one of them is captured.  That's why I use the {\tt
kqk.htb} tablebase.  Notice that I use it twice, depending on which
queen is being captured.  The {\tt kqk.htb} tablebase has a white
queen in it, and the {\tt colors="invert"} option to the {\tt
futurebase} element handles the case where the white queen is captured
and we're left with a black queen on the board.

You'll notice also that a four piece tablebase takes a good bit longer
to compute than a three piece one.

Oh, and I suppose having generating all of these tablebases, you now
want to query them, huh?

You do that using the probe (-p) option, followed by a list of
tablebases.  Since we've got a small collection of simple tablebases,
it's easiest to just load them all, like this:

\begin{verbatim}
C> ./hoffman -p *.htb
4 piece Nalimov tablebases found
Loading 'kk.htb'
Loading 'knk.htb'
Loading 'knkn.htb'
Loading 'knkp.htb'
Loading 'kpk.htb'
Loading 'kpkp.htb'
Loading 'kpkq.htb'
Loading 'kqk.htb'
Loading 'kqkn.htb'
Loading 'kqkp.htb'
Loading 'kqkq.htb'
Loading 'kqkr.htb'
Loading 'krk.htb'
Loading 'krkn.htb'
Loading 'krkp.htb'
Loading 'krkr.htb'
FEN? 8/8/8/8/p7/8/1P4k1/2K5 b
FEN 8/8/8/8/p7/8/1P4k1/2K5 b - -
Index 12658437
Draw

Nalimov score: DRAW
   g2h2    White moves and wins in 22
   g2f2    White moves and wins in 25
   g2g3    White moves and wins in 25
   g2g1    White moves and wins in 25
   g2f3    Draw
   g2h3    White moves and wins in 22
   g2f1    White moves and wins in 25
   g2h1    White moves and wins in 21
   a4a3    White moves and wins in 16
FEN or move? g2f3
FEN 8/8/8/8/p7/5k2/1P6/2K5 w - -
Index 12659332
Draw

Nalimov score: DRAW
   c1d1    Draw
   c1b1    Draw
   c1c2    Draw
   c1d2    Draw
   b2b3    Draw
   b2b4    Draw
FEN or move? b2b4
FEN 8/8/8/8/pP6/5k2/8/2K5 b - b3
Index 12593797
Draw

Nalimov score: DRAW
   f3g3    White moves and wins in 15
   f3e3    White moves and wins in 15
   f3f4    White moves and wins in 23
   f3f2    White moves and wins in 15
   f3e4    Draw
   f3g4    White moves and wins in 15
   f3e2    White moves and wins in 15
   f3g2    White moves and wins in 15
   a4xb3   Draw
FEN or move?
baccala@debian ~/src/endgame$
\end{verbatim}

At the ``FEN?'' prompt you want to enter a chess position in FEN
notation (you can leave off the castling rights and en passant square
if you want).  The program spits back its evaluation of the position
(if it has one), along with a list of moves and how they evaluate.  It
has a history feature, so once you've typed a FEN position in once, if
you end the program with a CNTL-D and not a CNTL-C, it will save
everything to a history file, and you can retrieve it again on a later
run using the up arrow key.

After you've put a FEN position in, you get a ``FEN or move?'' prompt,
which allows you to enter moves and thus step forward in the game.
The move parser isn't very smart right now; you can make illegal moves
pretty easily and there's no way to back up.  The quirkiest thing at
the moment is that if you want to promote, you need to specify EXACTLY
the piece you're promoting into; ``b7b8=Q'' is radically different from
``b7b8=q''!!

{\tt genctlfile.pl} is a Perl script that creates control files for
ordinary tablebases, though actually generating larger tablebases can
be quite demanding of a computer.  Hoffman has been used to generate a
complete set of five-piece tablebases, as well as a few sixes.

Hoffman can thus duplicate much of the functionality of the Nalimov
programs, but that's not all it can do.  It can also duplicate much of
the functionality of Eiko Bleicher's Freezer ({\tt
http://www.freezerchess.com/}).  Aside from Freezer's nice GUI, the
only thing Hoffman lacks is the ability to use Nalimov tablebases as
futurebases, which matters because 6 piece Nalimov tablebases are
available on-line, while only 5 piece tablebases are currently
available for Hoffman.  For example, here is a 1901 composition by
Lasker and Reichhelm, used as a demo on the Freezer website:

\fenboard{8/k7/3p4/p2P1p2/P2P1P2/8/8/K7 w - - 0 1}
\showboard

\begin{figure}
\hrule\ 
{\small\begin{verbatim}
<?xml version="1.0"?>
<!DOCTYPE tablebase SYSTEM "http://www.freesoft.org/software/hoffman/tablebase.dtd">

<tablebase>
   <prune-enable color="white" type="concede"/>
   <prune-enable color="black" type="concede"/>

   <piece color="white" type="king"/>
   <piece color="black" type="king"/>
   <piece color="white" type="pawn" location="a4"/>
   <piece color="white" type="pawn" location="d4"/>
   <piece color="white" type="pawn" location="f4"/>
   <piece color="white" type="pawn" location="d5"/>
   <piece color="black" type="pawn" location="a5"/>
   <piece color="black" type="pawn" location="f5"/>
   <piece color="black" type="pawn" location="d6"/>

   <prune color="white" move="KxP" type="concede"/>
   <prune color="black" move="KxP" type="concede"/>

   <generation-controls>
      <output filename="lasker1901.htb"/>
   </generation-controls>
</tablebase>
\end{verbatim}}
\hrule
\caption{\tt lasker1901.xml}
\end{figure}

The Hoffman XML configuration for this problem (Figure 5), is in the
{\tt xml} directory as {\tt lasker1901.xml}.  Several new
features in the configuration should be apparent.

First, note the new {\tt location} parameter that can be specified for
a piece to nail it down to a particular spot.  You can actually
specify a list of multiple squares, such as ``{\tt a4 a5 a6 a7}'';
we'll see an example of this later.

Note also the disappearance of the {\tt futurebase} element; it's been
replaced by a pair of {\tt prune} statements.  The program pretty much
has to have either {\tt futurebase} or {\tt prune} statements in order
to figure out how to handle things like captures.  In this case, we
don't use futurebases at all (so this is a stand-alone analysis), and
instead tell Hoffman to regard any pawn capture as an immediate
victory for the capturing side (that's the {\tt concede} part).

Finally, let me point out the {\tt prune-enable} elements at the
beginning of the {\tt tablebase} section.  They don't do very much,
but they're very important because the program will refuse to process
this tablebase without them.  They're there because it's easy to lose
track of pruning statements if they get buried inside futurebases.
You could easily use a pruned futurebase to build a tablebase that
wouldn't be accurate, but that also wouldn't be apparent just from
looking at the tablebase configuration.  That's why the {\tt
prune-enable} elements are there.  The program will refuse to allow
any pruning statement, or pruned futurebase, unless there is a
matching {\tt prune-enable} at the beginning of the {\tt tablebase}
section.  This prevents ``hidden'' prunes from slipping into an
analysis by mistake.  As their name suggests, the {\tt prune-enable}
elements ``turn on'' pruning; the program won't prune unless they're there.

OK, so by now we know how to generate {\tt lasker1901.htb} ({\tt
hoffman -g lasker1901.xml}, right?), and we can probe the resulting
tablebase for the original problem position:

\begin{verbatim}
C> ./hoffman -p lasker1901.htb
Hoffman $Revision: 1.15 $ $Locker: baccala $
0 piece Nalimov tablebases found
Loading 'lasker1901.htb'
FEN? 8/k7/3p4/p2P1p2/P2P1P2/8/8/K7 w
FEN 8/k7/3p4/p2P1p2/P2P1P2/8/8/K7 w - -
Index 2368
White moves and wins in 14
Can't find Nalimov tablebase
   Ka1b1    White wins in 14
   Ka1a2    Draw
   Ka1b2    Draw
FEN or move?
C>
\end{verbatim}

So what's the point?  Can White win in 14 moves?  No.  White can {\it
capture a black pawn} in 14 moves, but only if he plays Kb1!  Any
other move would allow Black to ``draw'', i.e, to prevent White from
capturing a pawn without allowing Black to capture first.  In fact,
those other two moves allow Black to prevent White from capturing at
all, but it's important to note that this Hoffman analysis doesn't
guarantee that, since Black might be able to ``win'' by capturing a
pawn.  White might be able to capture a pawn, just not before Black
can.

{\bf Question:} What changes to the configuration would allow us to
prove that Black can, in fact, completely prevent White from making
any capture after a move like Ka2?

{\bf Answer:} Change the black {\tt prune-enable} and {\tt
prune} elements from {\tt concede} to {\tt discard}.  Now the analysis
will not consider any black captures, and the only kind of ``draw''
will be one that completely prevents white from making any kind of
capture.  There is one gotcha, however.  White might put black in a
position where his {\it has} to capture a pawn, but the new analysis
would regard this as a stalemate.  To avoid this problem, you should
also add a third pruning element: {\tt <prune color="white"
move="stalemate" type="concede"/>}

Here's another example, also from the Freezer website:

\fenboard{8/6k1/6p1/8/q7/4R3/5PK1/8 w - - 0 1}
\showboard

It's a fortress.  We're White, and we're out to show that we can draw
this position.  My Hoffman analysis is in the ``{\tt fortress*.xml}''
files in the {\tt xml} directory.  You need to have several of
the four-piece tablebases built in order to process them.  I'm not
going to copy them all into this document; I'll just ask you to study
them on your own while I point out some salient features.

Notice how {\tt futurebase} and {\tt prune} elements can be used
together in a single configuration file (in case you were wondering).
Notice how wildcards can be used in {\tt prune} statements.  Notice
how a piece can be restricted to more than one square.  Notice how
comments can be put into XML configurations.

The {\tt pawngen} element provides a different way of handling pawns
-- a table is made of all possible pawn configurations that can
arise from a given start.

The next example is position 67 from Reuben Fine's book ``Basic Chess
Endings''.  There are too many pieces in play for Nalimov (eight), and
the pawn captures are too complicated for Freezer.

\fenboard{8/2p5/3k4/1p1p1K2/8/1P1P4/2P5/8 w - - 0 1}
\showboard

\begin{figure}
\hrule\ 
{\small\begin{verbatim}
<?xml version="1.0"?>
<!DOCTYPE tablebase SYSTEM "http://www.freesoft.org/software/hoffman/tablebase.dtd">
<!-- Diagram 67 from Rubin Fine's "Basic Chess Endings" -->
<!-- Problem position 8/2p5/3k4/1p1p1K2/8/1P1P4/2P5/8 w -->

<tablebase>
   <prune-enable color="white" type="concede"/>
   <prune-enable color="black" type="concede"/>
   <dtm/>
   <piece color="white" type="king"/>
   <piece color="black" type="king"/>
   <pawngen white-pawn-locations="b3 c2 d3" black-pawn-locations="b5 c7 d5"/>
   <futurebase filename="kpkp.htb"/>
   <futurebase filename="kppk.htb"/>
   <futurebase filename="kppk.htb" colors="invert"/>
   <prune color="white" type="concede" move="P=?"/>
   <prune color="black" type="concede" move="P=?"/>
</tablebase>
\end{verbatim}}
\hrule
\caption{\tt fine67.xml}
\end{figure}

The XML configuration (Figure 6) is {\tt fine67.xml}, but this file
can not be processed directly by Hoffman, because the program requires
the number and color of pieces to be fixed.  A starting pawn position
with three pawns on each side can lead to pawn positions with anywhere
from zero to three pawns on each side, after captures.  Also, Hoffman
only handles captures using futurebases; there is no option for
inter-tablebase captures.  Attempting to process this file
directly produces an error message:

\begin{verbatim}
C> ./hoffman -o fine67.htb -g fine67.xml
Hoffman $Revision: 1.15 $ $Locker: baccala $
terminate called after throwing an instance of 'xmlpp::validity_error'
  what():  
Validity error:
Element pawngen does not carry attribute black-pawns-required
Element pawngen does not carry attribute white-pawns-required


\end{verbatim}

Instead of using Hoffman directly, we need to run the Perl script ``pawngen'' on
{\tt fine67.xml}.  Pawngen will
output a series of interlinked control files that Hoffman can process.
Obviously, considering all possible promotions in this case would
result in positions with multiple queens that would be impossible for
Hoffman to process.  Pawngen, however, does honor the {\tt prune}
elements.  For this example, pawngen ignored all promotion
possibilities (note the {\tt prune} elements) and created 10 control
files.  Run the actual Hoffman analysis by typing ``make''.
Both Perl (needed to run pawngen) and make are fairly standard on
Linux and other UNIX variants, but neither are included in the Hoffman
distribution.  On Microsoft systems, I've used the cygwin distribution
(which includes both programs) successfully.

``make'' runs Hoffman 10 times and produces 10
tablebases.  Once this is done, we use Hoffman to probe the tablebases:

\begin{verbatim}
C> ./hoffman -p fine67-*.htb
Hoffman $Revision: 1.15 $ $Locker: baccala $
Loading 'fine67-0-3.htb'
Loading 'fine67-1-2.htb'
Loading 'fine67-1-3.htb'
Loading 'fine67-2-1.htb'
Loading 'fine67-2-2.htb'
Loading 'fine67-2-3.htb'
Loading 'fine67-3-0.htb'
Loading 'fine67-3-1.htb'
Loading 'fine67-3-2.htb'
Loading 'fine67-3-3.htb'
Index or FEN? 8/2p5/3k4/1p1p1K2/8/1P1P4/2P5/8 w
Index 1080618 (fine67-3-3.htb)
White wins in 27
   Pc2c3    (755539) Draw
   Pc2c4    (632731) Black wins in 19
   Pb3b4    (928915) White wins in 27
   Pd3d4    (936139) Draw
   Kf5g5    (1080729) Black wins in 18
   Kf5f6    (1081517) Draw
   Kf5f4    (1079721) Draw
   Kf5g6    (1081627) Black wins in 22
   Kf5g4    (1079831) Black wins in 24
FEN 8/2p5/3k4/1p1p1K2/8/1P1P4/2P5/8 w - -
Index, FEN or move? 
\end{verbatim}

This analysis confirms Fine's analysis that b4 is White's only winning
move in this position, although the pruning in the Hoffman analysis
only guarantees that White can force a queen before Black can.  In
particular, d4 only draws because Black has the recourse b4 which
allows him to blockade the position.

\begin{figure}
\hrule\ 
{\small\begin{verbatim}
<?xml version="1.0"?>
<!DOCTYPE tablebase SYSTEM "http://www.freesoft.org/software/hoffman/tablebase.dtd">
<!-- Diagram 67 from Rubin Fine's "Basic Chess Endings" -->
<!-- Problem position 8/2p5/3k4/1p1p1K2/8/1P1P4/2P5/8 w -->

<tablebase>
   <prune-enable color="white" type="discard"/>
   <prune-enable color="black" type="concede"/>
   <dtm/>
   <piece color="white" type="king"/>
   <piece color="black" type="king"/>
   <piece color="white" type="queen"/>
   <pawngen white-pawn-locations="b3 c2 d3" black-pawn-locations="b5 c7 d5" white-queens-required="1"/>
   <futurebase filename="kqkp.htb"/>
   <futurebase filename="kqpk.htb"/>
   <prune color="white" type="discard" move="P=?"/>
   <prune color="black" type="concede" move="P*=?"/>
   <prune color="black" type="concede" move="?xQ"/>
</tablebase>
\end{verbatim}}
\hrule
\caption{\tt fine67a.xml}
\end{figure}

\begin{figure}
\hrule\ 
{\small\begin{verbatim}
<?xml version="1.0"?>
<!DOCTYPE tablebase SYSTEM "http://www.freesoft.org/software/hoffman/tablebase.dtd">
<!-- Diagram 67 from Rubin Fine's "Basic Chess Endings" -->
<!-- Problem position 8/2p5/3k4/1p1p1K2/8/1P1P4/2P5/8 w -->

<tablebase>
   <prune-enable color="white" type="discard"/>
   <prune-enable color="black" type="concede"/>
   <dtm/>
   <piece color="white" type="king"/>
   <piece color="black" type="king"/>
   <pawngen white-pawn-locations="b3 c2 d3" black-pawn-locations="b5 c7 d5"/>
   <futurebase filename="fine67a-*.htb"/>
   <futurebase filename="kpkp.htb"/>
   <futurebase filename="kppk.htb"/>
   <futurebase filename="kppk.htb" colors="invert"/>
   <prune color="white" type="discard" move="P=[NBR]"/>
   <prune color="black" type="concede" move="P=?"/>
</tablebase>
\end{verbatim}}
\hrule
\caption{\tt fine67b.xml}
\end{figure}

Of course, we'd really like to know if White can win, not just force a
queen.  A more sophisticated analysis (Figures 7 and 8) is {\tt fine67a.xml}
and {\tt fine67b.xml}.
Notice the {\tt white-queens-required} attribute on the {\tt
pawngen} element.  This attribute indicates that the white queen
in the analysis arose from a pawn being queened.  In particular,
in {\tt fine67a.xml}
we only consider pawn positions that arise after one white
pawn has queened.

{\tt fine67a.xml} analyses the position {\it after} a white pawn has
queened.  To analyse the original problem position, we feed these
results to {\tt fine67b.xml}, which is quite similar to the original
{\tt fine67.xml}, except that we no longer discard queen promotions
and include the {\tt fine67a} futurebases.

Pawngen, when run on {\tt fine67a.xml}, produces 8 control files, and
{\tt fine67b.xml} yields 10 control files.  When complete, its output
tablebases can be queried for the original position:

\begin{verbatim}
C> ./hoffman -p fine67[ab]*.htb
Hoffman $Revision: 1.15 $ $Locker: baccala $
Loading 'fine67a-0-2.htb'
Loading 'fine67a-0-3.htb'
Loading 'fine67a-1-1.htb'
Loading 'fine67a-1-2.htb'
Loading 'fine67a-1-3.htb'
Loading 'fine67a-2-0.htb'
Loading 'fine67a-2-1.htb'
Loading 'fine67a-2-2.htb'
Loading 'fine67b-0-3.htb'
Loading 'fine67b-1-2.htb'
Loading 'fine67b-1-3.htb'
Loading 'fine67b-2-1.htb'
Loading 'fine67b-2-2.htb'
Loading 'fine67b-2-3.htb'
Loading 'fine67b-3-0.htb'
Loading 'fine67b-3-1.htb'
Loading 'fine67b-3-2.htb'
Loading 'fine67b-3-3.htb'
Index or FEN? 8/2p5/3k4/1p1p1K2/8/1P1P4/2P5/8 w
Index 1080618 (fine67b-3-3.htb)
White wins in 27
   Pc2c3    (755539) Draw
   Pc2c4    (632731) Black wins in 19
   Pb3b4    (928915) White wins in 27
   Pd3d4    (936139) Draw
   Kf5g5    (1080729) Black wins in 18
   Kf5f6    (1081517) Draw
   Kf5f4    (1079721) Draw
   Kf5g6    (1081627) Black wins in 22
   Kf5g4    (1079831) Black wins in 24
FEN 8/2p5/3k4/1p1p1K2/8/1P1P4/2P5/8 w - -
Index, FEN or move? 
\end{verbatim}

Again, we show a white win with {\tt b4}.  This time, however, we know
that this is actually a white {\it win} and not just a queening
solution.

%{\tt fine67b.xml} is a slight variant on this analysis that uses the
%{\tt basic} format type and an appropriate {\tt entries-format} to
%generate a {\it bitbase} --- a tablebase that records only win, lose,
%or draw information, without the distances we've seen so far in all of
%these examples.  A bitbase is more compact to store, faster to
%generate, and (thanks to the custom {\tt entries-format}) uses about
%half as much memory during generation.  Its generation is also more
%predictable, as there is no chance of overflowing its {\tt dtm} field,
%which would result in a fatal error fairly deep into its calculation.

%Pawngen currently accepts only a subset of Hoffman's XML syntax --- no
%{\tt location} attributes are allowed on any non-pawns, and pawns must
%have {\tt location} attributes specifying a specific starting square
%followed by a plus sign.  Despite these limitations, it is a key
%auxiliary tool that allows complicated pawn formations to be managed
%with a single control file.

Here's a position that Ivan Konobeev has studied extensively using
FinalGen, which can only conclude that Re8 is black's only hope,
without being able to establish it definitely as a win or a draw.

\fenboard{8/8/4r3/3k1K2/3p3P/8/6P1/7R b - - 0 1}
\showboard

The files {\tt ivand.xml}, {\tt ivane.xml}, and {\tt ivanf.xml}
contain a Hoffman analysis, with all promotion possibilities conceded
as wins.  Hoffman reaches a similar conclusion as FinalGen:

\begin{verbatim}
C> ./hoffman -p test/ivan[def]*.htb
Hoffman $Revision: 1.15 $ $Locker: baccala $
Loading 'test/ivand-2-0.htb'
Loading 'test/ivand-2-1.htb'
Loading 'test/ivane-2-0.htb'
Loading 'test/ivane-2-1.htb'
Loading 'test/ivanf-1-1.htb'
Loading 'test/ivanf-2-0.htb'
Loading 'test/ivanf-2-1.htb'
Index or FEN? 8/8/4r3/3k1K2/3p3P/8/6P1/7R b
Index 1876257633 (test/ivanf-2-1.htb)
Draw
   Pd4d3    (1766972960) White wins in 32
   Kd5c5    (1876257630) White wins in 6
   Kd5d6    (1876257642) White wins in 23
   Kd5c6    (1876257640) White wins in 6
   Kd5c4    (1876257620) White wins in 13
   Re6f6    (1876705520) White wins in 8
   Re6g6    (1877153408) White wins in 8
   Re6h6    (1877601296) White wins in 24
   Re6d6    (1875809744) White wins in 23
   Re6c6    (1875361856) White wins in 23
   Re6b6    (1874913968) White wins in 23
   Re6a6    (1874466080) White wins in 23
   Re6e7    (1879840736) White wins in 25
   Re6e8    (1883423840) Draw
   Re6e5    (1873122416) White wins in 35
   Re6e4    (1869987200) White wins in 25
   Re6e3    (1866404096) White wins in 25
   Re6e2    (1862820992) White wins in 25
   Re6e1    (1859685776) White wins in 6
FEN 8/8/4r3/3k1K2/3p3P/8/6P1/7R b - -
Index, FEN or move? 
\end{verbatim}

Specifically, all black moves other than Re8 allow white to force
a queen before black, while Re8 allows each side to prevent the
other from promoting.

The files {\tt ivana.xml}, {\tt ivanb.xml}, and {\tt ivanc.xml}
contain a Hoffman analysis with black promotions discarded, but since
white has the advantage in this position, this isn't a very
enlightening analysis.  More interesting would be to discard white
promotions, but an intuitive evaluation of the diagrammed position
suggests that white is unlikely to mate without promoting.

A more detailed analysis would require allowing queens into
the position along with the rooks.


See the Hoffman Reference Guide for more details on these and other
 Hoffman XML elements.

\end{document}
